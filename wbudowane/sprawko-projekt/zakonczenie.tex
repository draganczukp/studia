\section{Napotkane problemy}
\subsection{Dokładność wykrywania znaczników}

Z powodu braku plików z kalibracją kamery, program często "gubi" znaczniki, i nie jest w stanie
wykrywać ich ze stuprocentową skutecznością.

\subsubsection{Możliwe rozwiązanie}

Używając biblioteki OpenCV możliwe jest napisanie programu, który wygeneruje niezbędne informacje
kalibracyjne. Następnie można wczytać te informacje do programu i przekazać je jako parametry
funkcji \verb`aruco.DetectMarkers`.

\subsection{Dane nie są wysyłane przez port szeregowy}

Ze względu na brak czasu, nie udało się zaimplementować wysyłania wynikowych informacji przez port
szeregowy.

\subsubsection{Możliwe rozwiązania}

Możliwe są dwa rozwiązania. Pierwsze z nich nie wymaga modyfikacji kodu, a jedynie przekierowanie
standardowego wyjścia programu na port szeregowy za pomocą Unixowego operatora przekierowania
"\verb`>`". Drugie rozwiązanie to zamiana funkcji "\verb`print`"" w 28 linijce kodu na funkcję, 
która wyśle dane na port szeregowy.
