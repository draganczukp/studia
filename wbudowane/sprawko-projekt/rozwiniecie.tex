\section{Przebieg projektu}
\subsection{Zainstalowanie OpenCV}

Pierwszym krokiem było zainstalowanie biblioteki OpenCV.\@ Użyto do tego poniższego skryptu

\lstinputlisting{src/installopencv.sh}

\subsection{Przechwytywanie i wyświetlanie obrazu z kamery}
Po skończonej instalacji i kompilacji, następnym krokiem było przechwycenie obrazu z kamery i
wyświetlenie go w oknie. W tym celu użyto następującego kodu:

\lstinputlisting[language=Python]{src/camera.py}

Obraz nie jest w żaden sposób przetwarzany, i kod ten służy tylko do sprawdzenia, że kamera działa.

\subsection{Wykrywanie znaczników}

Mając obraz z kamery, można było zacząć wykrywać markery. Jako podstawy użyto kody z
poprzedniego punktu. Zmodyfikowano go tak, aby wykrywał znaczniki Aruco na przychwyconym
obrazie.

Na początku skryptu zadeklarowano dwie zmienne, wymagane przez OpenCV:

\begin{lstlisting}[language=Python]
aruco_dict  = aruco.getPredefinedDictionary(aruco.DICT_4X4_50)
parameters  = aruco.DetectorParameters_create()
\end{lstlisting}

Pierwsza zmienna to lista predefiniowanych znaczników, z kolekcji pięćdziesięciu znaczników o
wymiarach 4 na 4, druga to domyślne parametry
wyszukiwania znaczników.

Następnie ustawiono rozdzielczość, z jaką przechwytywany jest obraz z kamery

\begin{lstlisting}
cap = cv2.VideoCapture(0)
cap.set(cv2.CAP_PROP_FRAME_WIDTH, 1280)
cap.set(cv2.CAP_PROP_FRAME_HEIGHT, 720)
\end{lstlisting}

W głównej pętli dodano filtr, konwertujący przechwycony obraz na skalę szarości. Konwersja ta jest
wymagana przez OpenCV, gdyż znacząco zmniejsza to złożoność obliczeniową tego zadania.
\begin{lstlisting}
gray = cv2.cvtColor(frame, cv2.COLOR_BGR2GRAY)
\end{lstlisting}

W uzyskanym w ten sposób obrazie wyszukano wszystkiego, co przypomina znacznik.

\begin{lstlisting}[language=Python]
    corners, ids, rejected = aruco.detectMarkers(image=gray, dictionary=aruco_dict, parameters=parameters)

    if ids is not None:
        aruco.drawDetectedMarkers(frame, corners)

\end{lstlisting}

Następnie należy w jakiś sposób wykorzystać znalezione znaczniki. Poniższy fragment kodu oblicza
pozycję znacznika, a następnie wypisuje znalezione dane na standardowym strumieniu wyjścia:

\begin{lstlisting}[language=Python]
    if ids is not None:
        aruco.drawDetectedMarkers(frame, corners)
        for i in range(0, len(ids) ):
                corner = corners[i][0]
                x_sum = corner[0][0] + corner[1][0] + corner[2][0] + corner[3][0]
                x = x_sum /4
                normalised = (x - 640) / 640
                print(ids[i][0], -normalised)
\end{lstlisting}

Pozycja znacznika jest znormalizowana do zakresu $[-1, 1]$.

W każdej klatce, w której został znaleziony jakiś znacznik, na standardowym wyjściu wypisywana
jest informacja w formacie \verb`"[id_znacznika] [pozycja_x]"`

Skończony, działający kod wygląda następująco:
\lstinputlisting[language=Python]{src/detect.py}

