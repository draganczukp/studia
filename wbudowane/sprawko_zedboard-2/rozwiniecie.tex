\section{Przebieg ćwiczenia}
\subsection{Utworzenie nowego projektu}
Otwarto program Xilinx Platform Studio. W nowo otwartym okienku wybrano opcję "Base System
Builder Wizard (recommended)". W następnym oknie kliknięto na
przycisk "Browse" i wybrano folder, do którego powinien być zapisany projekt. W tym samym oknie
wybrano opcję "PLB System" i kliknięto OK.
W okienku "Board Selection" wybrano "I would like to create a new design option", potem kliknięto
"Next". Następnie wybrano informacje o używanej płytce:

\begin{center}
	\begin{tabular}{|l|l|}
		Board Vendor & Xilinx \\
		Board Name & Spartan-3E Starter Board \\
		Board Revision & D
	\end{tabular}
\end{center}
Kliknięto przycisk "Next", następnie upewniono się, że wybrana jest opcja "Single-Processor System"
i kliknięto "Next".

Ustawiono konfigurację procesora:

\begin{center}
	\begin{tabular}{|l|l|}
		Reference Clock Frequency & 50 MHz\\
		Processor Type			& MicroBlaze\\
		System Clock Frequency	& 50 MHz\\
		Local Memory			  & 8 KB\\
		Debug Interface		   & On-Chip H/W debug module\\
	\end{tabular}
\end{center}
\subsection{Konfiguracja urządzeń}

W oknie "Peripheral Configuration" skonfigurowano urządzenia peryferyjne znajdujące się na płytce

\subsubsection*{RS232\textunderscore DCE}

\begin{center}
	\begin{tabular}{|l|l|}
		RS232\textunderscore DCE & xps\textunderscore aurtlite \\
		Baud Rate & 115200 \\
		Data Bits & 8 \\
		Parity & None \\
		Use Interrupt & Odznaczone
	\end{tabular}
\end{center}
\subsubsection*{XPS GPIO (LEDs\textunderscore 8Bit):}

Odznaczone "Use Interrupt"

W oknie "Cache configuration" upewniono się, że nic nie jest zaznaczone. W oknie "Summary"
kliknięto "Finish"

\subsection{Analizowanie sprzętu}

Używając diagramu blokowego, wygenerowanego przyciskiem "Generate Block Diagram Image in
Project", oraz widoku "System Assembly View" znaleziono odpowiedzi na zadane pytania:

\subsubsection{List the name of the bus connection to the following peripherals:}

\begin{center}
\begin{tabular}{l l}
	mdm\textunderscore 0 & mb\textunderscore plb\\
	dlmb\textunderscore cntrl & dlmb\\
	RS232\textunderscore DCE & mb\textunderscore plb
\end{tabular}
\end{center}

\subsubsection{List the nets which are connected to the following ports}

\begin{center}
\begin{tabular}{l l}
	RS232\textunderscore DCE - RX& fpga\textunderscore 0\textunderscore RS232\textunderscore
	DCE\textunderscore RX\textunderscore pin\\
	RS232\textunderscore DCE - TX& fpga\textunderscore 0\textunderscore RS232\textunderscore
	DCE\textunderscore TX\textunderscore pin\\
	LEDs\textunderscore 8Bit - GPIO\textunderscore IO& No connection
\end{tabular}
\end{center}

\subsubsection{Select Addresses filter and list the address for the following instances:}

\begin{center}
\begin{tabular}{l l}
	RS232\textunderscore DCE - Base address& 0x84000000\\
	RS232\textunderscore DCE - High address& 0x8400FFFF\\
	LEDs\textunderscore 8Bit - Base address& 0x81400000\\
	LEDs\textunderscore 8Bit - High address& 0x8140FFFF\\
	dmlb\textunderscore cntlr - Base address& 0x00000000\\
	dmlb\textunderscore cntlr - High address& 0x00001FFF\\
	ilmb\textunderscore cntlr - Base address& 0x00000000\\
	ilmb\textunderscore cntlr - High address& 0x00001FFF\\
	DDR\textunderscore SDRAM - Base address& 0x8C000000\\
	DDR\textunderscore SDRAM - High address&  0x8FFFFFFF
\end{tabular}
\end{center}

\subsection{Wygenerowanie plików VHDL}

Za pomocą narzędzia "Hardware Generate Netlist" wygenerowano pliki VHDL opisujące obecną
konfigurację sprzętową.

Utworzone zostały następujące katalogi:

\begin{itemize}
	\item implementation
	\item synthesis
	\item hdl
	\item \textunderscore xps
\end{itemize}

\subsection{Generowanie aplikacji testującej pamięć}
W menu "Project" kliknięto opcję "Export Hardware Design to SDK". Kliknięto przycisk "Export \&
Launch SDK" z domyślnymi wartościami, a następnie wybrano folder "SKD\textunderscore Export" znajdujący się w
plikach projektu.

W menu "File" wybrano podmenu "New" a następnie wybrano opcję "Xilinx C Project". Z listy
"Select Project Template" wybrano szablon "Memory Test" i kliknięto "Next". Następnie kliknięto
"Finish", zostawiając domyślne opcje.

\subsection{Sprawdzanie za pomocą sprzętu}

Podłączono płytkę Spartan-3E do komputera oraz do zasilania według schematu w
zawartego w instrukcji.

Uruchomiono program Putty i ustawiono następujące opcje:
\begin{enumerate}
	\item W menu "Session" wybrano typ połączenia na Serial i ustawiono "Speed" na 115200
	\item W menu "Serial" ustawiono następujące wartości:
\end{enumerate}

\begin{center}
\begin{tabular}{|l|l|}
	Serial line to connect to & COM1\\
	Speed (baud) & 115200\\
	Data bits & 8\\
	Stop bits & 1 \\
	Parity & None\\
	Flow control & None
\end{tabular}
\end{center}

Następnie kliknięto "Open"

W menu "Xilinx Tools" wybrano pozycję "Program FPGA". W nowym okienku w kolumnie "ELF File to
Initialize in Block RAM" wybrano plik "memory\textunderscore test\textunderscore 0.elf"

Po kliknięciu przycisku "Program", plik .elf oraz plik system.bit zostały połączone i przesłane
do płytki. W oknie programu Putty wyświetlone zostały informacje o pamięci zainstalowanej na
płytce.
